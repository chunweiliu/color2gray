\section{Related Work}
\label{sec:related}

% Decolorization
% 1. Dimensionality reduction
% 2. Other method
% 3. Local
% 4. Global
% Isoluminant
% Gamut mapping

Decolorization can be view as a subset of dimensionality reduction problem.
Several algorithms have been proposed for compressing different types of data in general.
Principal Component Analysis (PCA) is a classic technique
of linear dimensionality reduction.
PCA function as decolorization by computing an ellipsoid to fit data in color space
and projecting points into the major axis of the ellipsoid.
In addition to linear methods, non-linear dimensionality reduction also served this task 
by fitting a more complex model in the color space. 
%as Teschioni~\etal's suggestion~\cite{Teschioni:1997:AMA}.
Both linear and non-linear model is sensitive to color space.
For example, RGB and CIEL*a*b lead totally different grayscale images in practice.
Despite of the complexity of the model using in this scenario, 
the main concern is that weather the color space reveal helpful hints
for them to explain data and perform mapping on it.
However, limited papers apply dimensionality reduction into their framework,
and even less of them discuss about the property of color space for this scheme. 
In a result that these methods were reported unstable for decolorziation~\cite{Gooch:2005:CSC}.

In recent years, several methods of decolorization have been proposed,
and they can be categorized in
%local~\cite{Bala:2004:SCT,Gooch:2005:CSC,Smith:2008:AGA,Lau:2011:CBC}
%and global~\cite{Rasche:2005:RIG,Grundland:2007:DFC,Kim:2009:RCV,Ancuti:2011:ESG} approaches.
Local operators aware of their neighborhood, they perform color mapping based on the spatial
relationship. 
In the other hand, global operators seek the same mapping to all pixels in the image
without losing significant contrast by careful design.

Bala and Eschbach~\cite{Bala:2004:SCT} combined luminance and chromatics through
high-pass filter which preserved edge information.
Although the filter-based operators is efficient, it can only handles the color differences
within a pixel width.
Gooch~\etal~\cite{Gooch:2005:CSC} proposed the technique that optimizes grayscale images
which satisfied the offsets of pixel-level chromatic differences.
Their study presented cheerful results on small scale image, but it is unable to scale up
by the limitation of a quartic order algorithm.
Smith~\etal~\cite{Smith:2008:AGA} used a two-step algorithm which 
globally mapped apparent lighting and then locally enhance the contrast by using
a pyramid-based decomposition of images.
In general, preserving information in neighborhood may help to enhance the contrast locally,
but they also might be suffer from distorting the order of illuminance in a wider view.
Moreover, they are computationally expensive as our reported in Table~\ref{tab:performance}.

On the other hand, 
Rasche~\etal~\cite{Rasche:2005:RIG} applied nonlinearly constraint optimization to seek
a global contrasts between all colors in the image instead of pixels.
They further refine their method by eliminating the computation~\cite{Rasche:2005:DPR}.
Grundland and Dodgson~\cite{Grundland:2007:DFC}
performed global color mapping with predominant component analysis.
The straightforward operator is efficient by Gaussian pairing sampling
which reduce the amount of comparing color differences.
By considering those comparison of color differences should not be handled in pixel level,
Lau~\etal~\cite{Lau:2011:CBC} proposed a cluster-based algorithm which manipulate the
operator on graphical model.
Rather than seeking a perfect optical match, Ancuti~\etal~\cite{Ancuti:2011:ESG}
argued the decolorization operator should focus on the salient regions in the picture.

Inspired by these methods, considering the advantages: 
(1) efficiency by computing on upper pixel level
(2) saliency by focusing on important region in the image,
we propose a global color mapping which preserves these
features by learning a linear projection through a sparse space with higher dimensionality.
We argue that the power of linear operator has been underestimated.
With a suitable feature transform, we preserve contrast and saliency without suffering
from potential isoluminants color in the input image.
As a global method, our algorithm avoid to create artifacts such as halos or 
broken segments in local methods.
In contrast with other global methods, we straightly preserve the most salient region
in the images to exploit the power of linear operator as demonstrated in 
Figure~\ref{fig:saliency}.
Different from Ancuti~\etal~\cite{Ancuti:2011:ESG}, our method default do not enhancing
the indistinct pattern as showed in Figure~\ref{fig:comparison}.


